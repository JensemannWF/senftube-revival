\documentclass[12pt]{article}
\usepackage{fontspec}
\usepackage{graphicx}
\usepackage{geometry}
\usepackage{hyperref}
\usepackage{microtype}
\usepackage{parskip}
\usepackage{xcolor}

\geometry{a4paper,margin=2.5cm}

\hypersetup{
  colorlinks=true,
  linkcolor=black,
  urlcolor=blue,
  pdftitle={Senftube-Revival},
  pdfauthor={Jens Buttenschön},
}

\setmainfont{TeX Gyre Pagella}

\title{\Huge \textbf{Senftube Revival}\\[1ex]
\large Eine Würzbewegung geht viral}
\author{\large Jens Buttenschön\\
\href{https://github.com/JensemannWF/senftube-revival}{github.com/JensemannWF/senftube-revival}}
\date{\small \today}

\begin{document}
\maketitle

\section*{Ad Olaf, aus Klein-Heidorn}
Wenn du das hier irgendwo siehst: Melde dich! Die Hälfte dieser Idee gehört dir. Damals, 1983, in der Ausbildung – mit deiner Senftube in der Blaumanntasche. Wir haben es nie vergessen.

\section*{Was ist ein Senftuber?}
Ein Senftuber ist jemand, der nicht redet, sondern \emph{drückt}. Eine Tube Senf in der Brusttasche – ganz selbstverständlich. Beim Frikadellenbrötchen vom Bäcker wird nicht diskutiert, da wird gegärtnert: Deckel auf, Senf rein, fertig.

In einer Welt voller Influencer braucht es jetzt: \textbf{Würzfluenza}. Menschen, die keine Werbung machen, sondern Geschmack. Die sich nicht schminken, sondern Frikadellen würzen. Die nicht über Konsum reden, sondern einfach mit dem Senf vorangehen.

\section*{Die Kampagne}
Wir schlagen eine virale Kampagne vor, die auf Einfachheit, Sichtbarkeit und Humor setzt:

\begin{itemize}
  \item \textbf{Senftube sichtbar}: Blaumanntasche, Hemdtasche, Hoodie – Hauptsache, die Tube schaut raus.
  \item \textbf{Fußgängerzonenaktion}: „Wer mit einer Senftube gesichtet wird, bekommt eine Prämie.“
  \item \textbf{Online-Kampagne}: Senftube als kultiger Hashtag – \#senftuber, \#würzfluenza, \#senfschüscher.
  \item \textbf{Stille YouTube-Parodie}: Keine direkte Nennung, aber klare Bildsprache – Senftuber als Antwort auf Content-Creator.
\end{itemize}

\section*{Senfschüscher und Würzphantasien}
Du kennst sie: Die jungen Leute, die an E-Shishas ziehen, als gäbe es nichts Schöneres. Und dann kommt da einer dazu – schraubt seelenruhig seine Senftube auf – und zieht einen tiefen Zug \emph{Senf}. Da sind alle platt.

Wir brauchen keine Ersatzstoffe mit künstlichem Qualm. Wir brauchen Würze! Die \textbf{Senfschüscher} sind da. Kult statt Rauch.

\section*{Warum das funktioniert}
Weil es echt ist. Weil man nichts kaufen muss, nichts verstehen muss, nichts erklärt bekommt. Nur einen Moment beobachten und lachen. Oder mitmachen.

Der Absatz an Tubensenf könnte durch so eine Kampagne sprunghaft steigen. Und Senf wird zur Stilfrage. Das gab es noch nie.

\section*{Lizenz und Angebot}
Dieses Konzept ist veröffentlicht unter der Creative Commons BY-NC-SA 4.0 Lizenz. Für kommerzielle Nutzung bitte mit uns Kontakt aufnehmen.

\bigskip
\textit{Kontakt und Originalkonzept:}\\
\texttt{https://github.com/JensemannWF/senftube-revival}

\end{document}